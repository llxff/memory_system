\documentclass[russian,utf8,nocolumnsxix,nocolumnxxxii,nocolumnxxxi,hpadding=10mm]{eskdtext}
\usepackage[russian]{babel}
\usepackage[T2A]{fontenc}
\usepackage{longtable}
\usepackage{graphicx}
\usepackage{mathtext}
\usepackage[numberright]{eskdplain}
\usepackage[utf8]{inputenc}
\usepackage{listings}
\usepackage{color}
\ESKDdepartment{МИНИСТЕРСТВО ОБРАЗОВАНИЯ И НАУКИ РФ}
\ESKDcompany{Университет управления «ТИСБИ»}
\renewcommand{\ESKDtitleFontIII}{\normalsize}
\renewcommand{\ESKDtheTitleFieldIII}{Специальность 09.03.01 «Информатика и вычислительная техника»}
\renewcommand{\ESKDtitleFontIV}{\LARGE}
\renewcommand{\ESKDtheTitleFieldIV}{КУРСОВАЯ РАБОТА}
\renewcommand{\ESKDtitleFontV}{\Large}
\renewcommand{\ESKDtheTitleFieldV}{НА ТЕМУ: «Разработка объектной программы для задачи моделирования подсистемы управления основной памятью»}
\renewcommand{\ESKDtheTitleFieldX}{Казань,~2017г.}
\renewcommand{\ESKDtheTitleFieldVIII}{
\null\hfill
\begin{minipage}{0.5\textwidth}
\normalsize ВЫПОЛНИЛ студент гр. ДЗИ 105Н:
\\Фомин~А.~В.
\\ПРОВЕРИЛ преподаватель:
\\Козин~А.~Н.
\end{minipage}\\}
\ESKDdate{2017/07/21}
\renewcommand{\thefootnote}{\arabic{footnote}}
\begin{document}
\maketitle
\newpage
\hyphenpenalty=1000
\exhyphenpenalty=1000
\sloppy

\linespread{1,4}
\tableofcontents

\newpage
\parindent=12.5mm
\linespread{1,5}
\section{Постановка задачи}

Разработать объектную программу для хранения и обработки данных о процессах и используемых ими страницах основной памяти в многозадачных ОС. Каждый процесс имеет уникальный идентификатор и рассматривается как набор страниц. В свою очередь каждая страница имеет уникальный номер и параметр «Состояние».

Разработка включает в себя определение необходимых объектов и описание их в виде классов, программную реализацию методов добавления и удаления процессов и страниц, всестороннее тестирование методов с помощью консольного (при разработке) и оконного (в окончательном варианте) приложения.

Для объединения процессов используется структура данных в виде адресного разомкнутого упорядоченного однонаправленного списка без заголовка. Для объединения страниц внутри процессов  используется очередь на основе обычного массива со сдвигом элементов.

\newpage
\parindent=12.5mm
\linespread{1,5}
\section{Описание используемых структур данных с алгоритмами выполнения основных операций}




\newpage
\parindent=12.5mm
\linespread{1,5}
\section{Краткие сведения об объектном подходе}

\newpage
\parindent=12.5mm
\linespread{1,5}
\section{Формализованное описание разработанных классов}

\newpage
\parindent=12.5mm
\linespread{1,5}
\section{Описание демонстрационного модуля с характеристикой использованных стандартных компонентов и списком реализованных обработчиков}

\newpage
\parindent=12.5mm
\linespread{1,5}
\section{Полный листинг программы с краткими комментариями}


\definecolor{dkgreen}{rgb}{0,0.6,0}
\definecolor{gray}{rgb}{0.5,0.5,0.5}
\definecolor{mauve}{rgb}{0.58,0,0.82}
\definecolor{light-gray}{gray}{0.25}

\lstdefinestyle{java}{
  language=Java,
  aboveskip=3mm,
  belowskip=3mm,
  showstringspaces=false,
  columns=flexible,
  basicstyle={\footnotesize\ttfamily},
  numberstyle={\tiny},
  numbers=left,
  keywordstyle=\color{blue},
  commentstyle=\color{dkgreen},
  stringstyle=\color{mauve},
  breaklines=true,
  breakatwhitespace=true,
  tabsize=3
}

\lstdefinestyle{xml}{
  language=XML,
  aboveskip=3mm,
  belowskip=3mm,
  showstringspaces=false,
  columns=flexible,
  basicstyle={\footnotesize\ttfamily},
  numberstyle={\tiny},
  numbers=left,
  keywordstyle=\color{blue},
  commentstyle=\color{dkgreen},
  stringstyle=\color{mauve},
  breaklines=true,
  breakatwhitespace=true,
  tabsize=3,
  morekeywords={AnchorPane, children, TreeView, TableView, placeholder, Label, columns, TableColumn, font, Font, Button, import}
}

\lstinputlisting[caption=Main, style=java,  inputencoding=cp1251]{src/main/java/com/github/llxff/Main.java}

\lstinputlisting[caption=MainController, style=java,  inputencoding=cp1251]{src/main/java/com/github/llxff/MainController.java}

\lstinputlisting[caption=MemoryStatus, style=java,  inputencoding=cp1251]{src/main/java/com/github/llxff/MemoryStatus.java}

\lstinputlisting[caption=ProcessTreeCell, style=java,  inputencoding=cp1251]{src/main/java/com/github/llxff/ProcessTreeCell.java}

\lstinputlisting[caption=MemoryPage, style=java,  inputencoding=cp1251]{src/main/java/com/github/llxff/domain/MemoryPage.java}

\lstinputlisting[caption=MemoryPagesQueue, style=java,  inputencoding=cp1251]{src/main/java/com/github/llxff/domain/MemoryPagesQueue.java}

\lstinputlisting[caption=Process, style=java,  inputencoding=cp1251]{src/main/java/com/github/llxff/domain/Process.java}

\lstinputlisting[caption=ProcessesList, style=java,  inputencoding=cp1251]{src/main/java/com/github/llxff/domain/ProcessesList.java}

\lstinputlisting[caption=VirtualMachine, style=java,  inputencoding=cp1251]{src/main/java/com/github/llxff/domain/VirtualMachine.java}

\lstinputlisting[caption=project.fxml, inputencoding=utf8, style=xml]{src/main/resources/fxml/project.fxml}

\newpage
\begin{thebibliography}{99}
\bibitem{01} Жигалев М.А.,  Коваленко И.Г.,  Саюшев В.А. «Организация и методика производственного обучения». Москва «Высшая школа» 1978 г.
\end{thebibliography}
\end{document}
